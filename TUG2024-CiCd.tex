\iffalse
{Oliver Kopp, Marcel~Kr\"uger, Marei~Peischl}
{Tutorial: Creation of \LaTeX\ documents using a cloud-based pipeline}
{This will be a tutorial on using git as well as GitHub\slash GitLab
pipelines for \LaTeX\ document development. The duration is expected to
be two hours or more, depending on the final schedule.

Most \LaTeX{} users compile on the local machine or using online
services like Overleaf. There also is a third option: build servers.
That option is especially interesting when one versions the project
using git. This tutorial will show you how to get from an installation
of git to a \LaTeX\ document compiled by a build server. At first,
concepts of git and GitHub\slash GitLab will be explained. This includes
the concept of branches, pushes, automated checks on the build server,
and collaborative work using pull\slash merge requests. After that,
there will be time for practical exercises and questions.

Participants are asked to create a GitHub or GitLab account before the
start, but we will also show how to run a pipeline locally so this is
not a strict requirement.}
\fi

\documentclass[final]{ltugboat}
\def\ldots{\tubdots\allowbreak}

\usepackage{Peischl_expl3tut-config}
\usepackage{microtype}
\usepackage{graphicx}
\usepackage[hidelinks,pdfa]{hyperref}
\usepackage[autostyle]{csquotes}


%\usepackage{biblatex}

\usepackage{tabularx}
%%% Start of metadata %%%

\title[TUG2024: LaTeX CI/CD]{Tutorial:\newline Creation of \LaTeX\ documents using a cloud-based pipeline}
\date{2024-07-19}
\author{Oliver Kopp \and Marcel~Krüger \and Marei~Peischl}

% repeat info for each author; comment out items that don't apply.
\author{Marei Peischl}
\address{Gneisenaustr. 18\\20253 Hamburg\\Germany}
\netaddress{marei (at) peitex dot de}
\personalURL{https://peitex.de}
%\ORCID{0}
% To receive a physical copy of the TUGboat issue, please include the
% mailing address we should use, as a comment if you prefer it not be printed.
\author{Marcel Krüger}
%\address{Gneisenaustr. 18\\20253 Hamburg\\Germany}
%\netaddress{marei (at) peitex dot de}
%\personalURL{https://peitex.de}
\author{Oliver Kopp}
%\address{Gneisenaustr. 18\\20253 Hamburg\\Germany}
%\netaddress{marei (at) peitex dot de}
%\personalURL{https://peitex.de}

%%% End of metadata %%%
%\addbibresource{dante-expl3tut.bib}       % xampl.bib comes with BibTeX}

\newif\ifTUGboatPrint

\TUGboatPrinttrue

%black & white mode for printing
\ifTUGboatPrint
\usemintedstyle{bw}
\AtBeginEnvironment{minted}{\let\textit\textsl\let\itshape\slshape}
\AtBeginEnvironment{tcolorbox}{\let\textit\textsl\let\itshape\slshape}
\tcbset{
	MintedFunctionframe/.append style={
		before upper={\let\textit\textsl\let\itshape\slshape}
	}
}
%These are the background colors of the boxes
\colorlet{CommentColBack}{black!2}
\colorlet{ListingColBack}{black!8}
\colorlet{TextColBack}{black!4}
\colorlet{TitleColBack}{white}
\fi

\usepackage{hologo}
\expandafter\def\csname HoLogo@TeX Live\endcsname{
	\TeX\,Live
}


%TODO acro 
% cici = continous ingegration and continous delivery
\begin{document}

\maketitle

Everything moved or is moving towards the cloud.
\LaTeX{} is already there for about 10 years and today it's quite common to use a web editor for collaboration and local compilation became the \enquote{nerdy} way.
But there also is a third variant to compile documents which can be used also to improve package development and in general the stability of \LaTeX{}:
Adapting DevOps methods, like \acro{CI/CD} for the workflow and compiling using automated workflows. 

\section{Why should I?}
Having some working workflows usually makes people to not think about changing anything.
So there have to be reasons why it might be worth reading this article or even integrating the mechanisms into projects.

Originally the idea was to folliw the current state of OpenSource development and open a door for contributors in the development process, 
For example the \LaTeX{} Project is currently creating a lot of user interaction via their GitHub projects~\cite{latex3-github} and also taking contributions from which the whole \TeX/\LaTeX{} community will profit.

But the advantages of these methods go even further\footnote{We focus on some aspects as this could probably male an article by itself.}

\subsection{Works for me?!}
Sometimes compilation failes on some systems, but doesn't on others. 
This may have tons of reasons so having some pipeline configured will show if this a machine specific issue or a general one. 

\subsection{Compatibility and regression testing}
It's possible to run the workflow on multiple TeX distributions or versions.
This can be used to know if there are any issue with some package update before the actual update where downgrading may be complicated\footnote{That's another issue to address … but a differen story.}

Additionally you can also use it to check backwars compatibility e.\,g. if one collaborator is using a Debian Stable which is stuck on some outdated version. 
As mentioned the \LaTeX{} Project Team is already using this techniques and even provides functionality for regression testing within their build system \enquote{l3build} \cite{l3build}.

Using this as a package developer enables a general interface to be used for regression testing.
This allows to avoid some of the bugs which otherwhise would be published and should be found by a user. 
Also it can be used to avoid inconsitent structures, e.\,g. one of the authors recently found that tons of packages and files within \TeXLive{} don't have a proper version number set within the code.

\section{Structure of this tutorial}

The idea is to introduce readers to the basics of setting up automated workflows on GitLab, GitHub or Forgejo.
As all plattforms are git based we expect the project to be some kind of git repository.
In case the reader does not yet use git, there is a little bit of information attached to this tutorial.
Using that it would be possible to use git without even noticing as it's attached to the autosave function of an editor. 

For the workshop at TUG2024 there were demo projects prepared which can be used as a template.
An overview of all is shown in table \ref{tab:demo-repos}.

\begin{table}
\caption{Overview over the template repositories prepared including direct links}
\begin{tabularx}{\linewidth}{llX}
Name & Setup Info & Description\\

\end{tabularx}
\end{table}



The tutorial at TUG2024 was the result of some discussion after DANTE2024.
Oliver Kopp had offered a workshop there introducing the participants to use GitHub Actions \cite{cicd-workshop-DANTE2024}. 
This was a bit too specific and missed to show the full power of using automated workflows, so the tutorial at TUG was an extension of this.
Within this article the idea now is to have a written tutorial instructing users to take their first steps.
The source will be published as well as we want this resource to be improved as much as possible.
So any kind of feedback is very welcome.

This tutorial mainly addresses two groups of users:

\begin{enumerate}
\item Authors which focus on typesetting actual content. This includes collaborations.
\item Package or template developers, who provide their work to be used by the first group.
\end{enumerate}
The second group obviously can also  use workflows of the first e,g, for typesetting documentation.
So developers usually use an extended version of the the setup provided for the authors. 




\section{Version controlling}
It's not enforced but highly recommended to use some version vontrol system if you want to get started using \acro{CI/CD} methods.
Within this tutorial git \cite{git} will be used.
To get started with this one does not have any idea about how to use git properly with software development.
In case it's not used yet there are editors supporting git to be used with the autosave functionality.
This would already be enough for using the setups described in ths article.

%\printbibliography
\def\url{\tbsurl}
\SetBibJustification{\raggedright \advance\itemsep by 1pt plus1pt minus1pt }
\bibliographystyle{tugboat}
\bibliography{tug2024-cicd}
\makesignature




\end{document}
