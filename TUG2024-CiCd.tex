\iffalse
{Oliver Kopp, Marcel~Kr\"uger, Marei~Peischl}
{Tutorial: Creation of \LaTeX\ documents using a cloud-based pipeline}
{This will be a tutorial on using git as well as GitHub\slash GitLab
pipelines for \LaTeX\ document development. The duration is expected to
be two hours or more, depending on the final schedule.

Most \LaTeX{} users compile on the local machine or using online
services like Overleaf. There also is a third option: build servers.
That option is especially interesting when one versions the project
using git. This tutorial will show you how to get from an installation
of git to a \LaTeX\ document compiled by a build server. At first,
concepts of git and GitHub\slash GitLab will be explained. This includes
the concept of branches, pushes, automated checks on the build server,
and collaborative work using pull\slash merge requests. After that,
there will be time for practical exercises and questions.

Participants are asked to create a GitHub or GitLab account before the
start, but we will also show how to run a pipeline locally so this is
not a strict requirement.}
\fi

\documentclass[final]{ltugboat}
\def\ldots{\tubdots\allowbreak}

\usepackage{Peischl_expl3tut-config}
\usepackage{microtype}
\usepackage{graphicx}
\usepackage[hidelinks,pdfa]{hyperref}
\usepackage[autostyle]{csquotes}


%\usepackage{biblatex}

\usepackage{tabularx}
%%% Start of metadata %%%

\title[TUG2024: LaTeX CI/CD]{Tutorial:\newline Creation of \LaTeX\ documents using a cloud-based pipeline}
\date{2024-07-19}

% repeat info for each author; comment out items that don't apply.
\author{Marei Peischl}
\address{Gneisenaustr. 18\\20253 Hamburg\\Germany}
\netaddress{marei (at) peitex dot de}
\personalURL{https://peitex.de}
%\ORCID{0}
% To receive a physical copy of the TUGboat issue, please include the
% mailing address we should use, as a comment if you prefer it not be printed.
\author{Marcel Krüger}
%\address{Gneisenaustr. 18\\20253 Hamburg\\Germany}
%\netaddress{marei (at) peitex dot de}
%\personalURL{https://peitex.de}
\author{Oliver Kopp}
%\address{Gneisenaustr. 18\\20253 Hamburg\\Germany}
%\netaddress{marei (at) peitex dot de}
%\personalURL{https://peitex.de}

%%% End of metadata %%%
%\addbibresource{dante-expl3tut.bib}       % xampl.bib comes with BibTeX}

\newif\ifTUGboatPrint

\TUGboatPrinttrue

%black & white mode for printing
\ifTUGboatPrint
\usemintedstyle{bw}
\AtBeginEnvironment{minted}{\let\textit\textsl\let\itshape\slshape}
\AtBeginEnvironment{tcolorbox}{\let\textit\textsl\let\itshape\slshape}
\tcbset{
	MintedFunctionframe/.append style={
		before upper={\let\textit\textsl\let\itshape\slshape}
	}
}
%These are the background colors of the boxes
\colorlet{CommentColBack}{black!2}
\colorlet{ListingColBack}{black!8}
\colorlet{TextColBack}{black!4}
\colorlet{TitleColBack}{white}
\fi

\usepackage{hologo}
\expandafter\def\csname HoLogo@TeX Live\endcsname{
	\TeX\,Live
}

\newcommand*{\TeXLive}{\acro{\TeX\,Live}}

\newcommand*{\file}[1]{\texttt{#1}}

\usepackage{fontspec}

\directlua{luaotfload.add_fallback
("emojifallback",
	{"NotoColorEmoji:mode=harf"}
)}

\setmonofont{Latin Modern Mono}[RawFeature={fallback=emojifallback}]

%TODO acro
% cici = continous ingegration and continous delivery
\begin{document}

\maketitle

Everything moved or is moving towards the cloud.
\LaTeX{} is already there for about 10 years and today it's quite common to use a web editor for collaboration and local compilation became the \enquote{nerdy} way.
But there also is a third variant to compile documents which can be used also to improve package development and in general the stability of \LaTeX{}:
Adapting DevOps methods, like \acro{CI/CD} for the workflow and compiling using automated workflows.

\section{Why should I?}
Having some working workflows usually makes people to not think about changing anything.
So there have to be reasons why it might be worth reading this article or even integrating the mechanisms into projects.

Originally the idea was to folliw the current state of OpenSource development and open a door for contributors in the development process,
For example the \LaTeX{} Project is currently creating a lot of user interaction via their GitHub projects~\cite{latex3-github} and also taking contributions from which the whole \TeX/\LaTeX{} community will profit.

But the advantages of these methods go even further\footnote{We focus on some aspects as this could probably male an article by itself.}

\subsection{Works for me?!}
Sometimes compilation failes on some systems, but doesn't on others.
This may have tons of reasons so having some pipeline configured will show if this a machine specific issue or a general one.

\subsection{Compatibility and regression testing}
It's possible to run the workflow on multiple TeX distributions or versions.
This can be used to know if there are any issue with some package update before the actual update where downgrading may be complicated\footnote{That's another issue to address … but a differen story.}

Additionally you can also use it to check backwards compatibility e.\,g. if one collaborator is using a Debian Stable which is stuck on some outdated version.
As mentioned the \LaTeX{} Project Team is already using this techniques and even provides functionality for regression testing within their build system \enquote{l3build} \cite{l3build}.

Using this as a package developer enables a general interface to be used for regression testing.
This allows to avoid some of the bugs which otherwhise would be published and should be found by a user.
Also it can be used to avoid inconsitent structures, e.\,g. one of the authors recently found that tons of packages and files within \TeXLive{} don't have a proper version number set within the code.

\section{Structure of this tutorial}

The idea is to introduce readers to the basics of setting up automated workflows on GitLab, GitHub or Forgejo.

This tutorial mainly addresses two groups of users:

\begin{enumerate}
\item Authors which focus on typesetting actual content. This includes collaborations.
\item Package or template developers, who provide their work to be used by the first group.
\end{enumerate}

The second group obviously can also  use workflows of the first e,g, for typesetting documentation.
So developers usually use an extended version of the the setup provided for the authors.

As all used platforms usually are used to be used with git we expect the project to be some kind of git repository.
In case the reader does not yet use git, there is a little bit of information attached to this tutorial.
Using that it would be possible to use git without even noticing as it's attached to the autosave function of an editor.

As the \LaTeX{} project is using GitHub this is how we are going to start with detailed explanation and afterwards will create a matching setup on the other platforms.
There have been example repositiories created which can be used as a template.

\section{Pipelines for document authors}

The most setups are somehow based on Docker.
The basic structure was described in \cite{islandoftex-docker} including options to run a container locally.
The so-called runners for the pipeline would do the same. They automatically start a container wich a specific image and are able to run scripts in there.

In general a document building pipeline has to consist of the following steps:

\begin{itemize}
\item Setup the container
\item checkout the repository
\item run the compilation command
\end{itemize}

GitHub is adding some abstraction layer for those tasks.
So it's possible to find actions to do multiple steps at once, e.g. the latex-action \cite{latex-action} is starting a container inside the container to run latexmk. Or just addoing commands like in a shell-script.

The configuration for GitHub is done by creating a yaml file within the repositories subdirectory \file{.github/workflows/build.yml}.

%\begin{listing}
\inputminted[linenos]{yaml}{examples/hello-world.yml}
%\caption{ \file{.github/workflows/build.yml} directly using a LaTeX action}
% \label{list:hello-world}
%\end{listing}

Listing \ref{list:hello-world} shows a minimal action config:

\begin{description}
\item[name:] of the workflow.
This is important if a project contains multiple workflows.

\item[on:] This directive selects when the workflow should be started.
This setting means it will be started on any push. So whenever the repository on the server receives an update.

\item[jobs] contans a list of jobs to be run after each other.
For example it's quite common to have one for testing and one for deployment.
In this example it only contains one job called \enquote{action-test}.

\item[runs-on:] the value corresponds to a runner setup.
Runners are the machines to actually run the workflow.
It doesn't have to be the same server as the one where the repository is hosted.
In the example \enquote{ubuntu-latest} is chosen.
This is one of the provided runners by GitHub which is based on ubuntu.
It includes nodejs and some tooling to simplify the work using predefned actions.
A full list of the runners and detailed description of the images can be found at cite{github-hosted-runners}.

\item[steps:] This is what the workflow should actually do.
As one can see it's possible to directly enter bash code in there.
This example is only creating some output and should therefore run without any issues.
\end{description}

This action has nothing to do with the tasks which was planned to do.
So it's necessary to extend the steps.

\inputminted[linenos,firstline=5]{yaml}{examples/latex-basic.yml}

\begin{description}
\item[container:] By this way it's possible to choose a different container image. Here it's the \TeXLive{} container provided by the Island of \TeX.
\item[steps:] The second step is structured as before and just received an extra name. This command is directly run. The first step is different. It uses the |uses:| directive. This is referring to a predefined GitHub action\cite{action/checkout}.

These can be used to combine multiple tasks in one step. For example using the action is automatically resolving the login process and takes care of choosing the directories correctly. After this step the shell to run the other command is  inside the repositories root directory.
\end{description}

As latexmk automatically tries to detect how many runs are necessary it's a great choice.
Additionally it will automatically run onto all *.tex files in the directory. Any other setup could either be configured using the workflow run or a local latexmkrc file. Details on the configuration can be found in the documentation \cite{latexmk}.




%\printbibliography
\def\url{\tbsurl}
\SetBibJustification{\raggedright \advance\itemsep by 1pt plus1pt minus1pt }
\bibliographystyle{tugboat}
\bibliography{tug2024-cicd}
\makesignature




\end{document}
